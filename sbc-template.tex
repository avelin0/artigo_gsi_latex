\documentclass[12pt]{article}

\usepackage{sbc-template}

\usepackage{graphicx,url}

\usepackage[brazil]{babel}   
%\usepackage[latin1]{inputenc}  
\usepackage[utf8]{inputenc}  

\sloppy

\title{Gestão de Segurança da Informação \\
(GSI)}

\author{Bruno Avelino de Araújo Oliveira, Rodrigo Nascimento Estolano da Silveira, André Luiz de Mesquita Melo }

\address{Seção de Engenharia da Computação -- Instituto Militar de Engenharia
  (IME)\\
  Caixa Postal 22291-270 -- Rio de Janeiro -- RJ -- Brazil
  \email{avel.bruno@gmail.com, rodrigonsilveira@hotmail.com, andreluiz.melo@outlook.com
}
}
\begin{document} 

\maketitle

\begin{abstract}
    This article aims to analyze of Information Security Management concerning Information Security Policy, Information Security Management Systems and Information Security Standards.
\end{abstract}
     
\begin{resumo} 
Este artigo tem como objetivo fazer uma análise da
Gestão da Segurança da Informação quanto à Politica de Segurança da Informação, Sistemas de Gestão de Segurança da Informação e quanto às Normas de Segurança da Informação presentes.


\end{resumo}

\section{Introdução}

\section{PSI}
\subsection{Definição}
A Política de Segurança da Informação(PSI) é um conjunto de regras, diretrizes e procedimentos, reunidos em um documento, que
têm como objetivo preservar informações importantes de uma organização. 
O PSI define um padrão de segurança a ser seguido pelo corpo de funcionários, usuários internos e externos e 
também no escopo tecnológico.

De acordo com a norma ABNT NBR ISO/IEC 17799:2005 o objetivo deste documento é
"Prover uma orientação e apoio da direção para a segurança da informação de acordo com os requisitos do negócio e com as leis e regulamentações relevantes. Convém que a direção estabeleça uma política clara, alinhada com os objetivos do negócio e demonstre apoio e comprometimento com a segurança da informação por meio da publicação e manutenção de uma política de segurança da informação para toda a organização".

A Política de Segurança da Informação é importante para que haja uma estratégia definida para proteção de ativos.
É necessário que ela contenha as responsabilidades das funções relacionadas a segurança e indentifique as principais ameaças, riscos e impactos envolvidos.

Alguns exemplos de diretrizes que podem estar presentes na PSI são políticas de backup, utilização de rede, instalação de softwares, etc.
Isso inclui proibição de uso de software sem licença, proibição de envio de material sexualmente explícito usando a rede da empresa, cuidados necessários com senhas.

\subsection{Elaboração e composição}

A elaboração da PSI precisa envolver a alta gerência, os gerentes de segurança da informação, gerentes e proprietários
de sistemas e principalmente os usuários. Ela deve responder o que significa segurança da informação,
porque os usuários devem se preocupar com isso e quais os objetivos da segurança da informação. 

A elaboração contém as seguintes fases:
\begin{itemize}
  \item Estruturar o Comitê de Segurança;
  \item Definir Objetivos;
  \item Realizar Entrevistas e Verificar a Documentação Existente;
  \item Elaborar o Glossário da Política de Segurança;
  \item Estabelecer Responsabilidades e Penalidades;
  \item Preparar o Documento Final da PSI;
  \item Oficializar a Política da Segurança da Informação;
  \item Sensibilizar os Colaboradores.
\end{itemize}

A Politica de Segurança da Informação se sustenta por diretrizes, normas e procedimentos.
As diretrizes se baseiam na missão e na visão da empresa para formar regras de nível estratégico.
Já as normas se aplicam a todos e são elaboradas de forma genérica, mas devem ser elaboradas com foco em um assunto específico como controle de acesso e uso de rede.
Os procedimentos são orientações e instruções operacionais e é importante que haja registro de que são efetivamente cumpridas.

Os produtos da documentação são a carta do presidente, diretrizes de segurança de informação,
normas gerais de segurança da informação e exemplos de procedimentos e técnicas operacionais e instruções técnicas.

\subsection{Fatores para o Sucesso}
É essencial para o sucesso da Política de Segurança da Informação que
ela seja analisada e sofra evolução para atender a novas conjunturas e mudanças de cenário.
De acordo com a ISO 27002, deve-se assegurar sua contínua pertinência, adequação e eficácia.
Deve-se também analisar seu desempenho para confirmar que esteja cumprindo sua função. Além disso é importante que
ela seja clara e concisa, ou seja, esteja escrita de forma formal e acessível e não contenha informações desnecessárias.

É preciso também garantir sua difusão na empresa, com divulgação e treinamentos com usuários, clientes e fornecedores.
A cultura da empresa também é importante, pois se os funcionários valorizarem a segurança da informação,
eles levarão mais a sério as normas e diretrizes.

Podem atrapalhar o sucesso da PSI a falta de recursos, inclusive humanos, 
ferramentas e orçamentos adequados e a falta de consciência sobre a política e sua importância.
Os usuários devem estar cientes das ameaças e equipados para lidar com elas.

\section{SGSI}


\section{Normas}

\section{Conclusão}


\newpage

\bibliographystyle{sbc}
\bibliography{sbc-template}

\end{document}
